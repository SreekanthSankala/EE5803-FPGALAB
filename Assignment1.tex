

\documentclass{article}

\usepackage{karnaugh-map}
\usepackage[utf8]{inputenc}
\usepackage{color}
\usepackage{comment}

\title{Assignment 1 | FPGA Lab}
\author{Sankala Sreekanth}
%\date{January 2022}

\begin{document}

\maketitle

\section{Question}
Reduce the following Boolean expression to its simplest form using K-Map
\begin{equation}
    F(X,Y,Z,W) = \sum (0,1,6,8,9,10,11,12,15)
\end{equation}

\section{Solution}


Represent the given Boolean expression in K-Map and follow the K-Map rules to reduce the given Boolean form to simplest form 



\begin{karnaugh-map}[4][4][1][$ZW$][$XY$]
        \minterms{0,1,6,8,9,10,11,12,15}
        \maxterms{2,3,4,5,7,13,14}
        \implicant{12}{8}
        \implicantedge{0}{1}{8}{9}
        \implicant{8}{10}
        \implicant{15}{11}
        \implicant{6}{6}
\end{karnaugh-map}

\begin{equation}
    F(X,Y,Z,W) = \overline{Y} . \overline{Z} +  X . \overline{Y} +  X . \overline{Z}. \overline{W} +  X . Z . W +  \overline{X} . Y . Z . \overline{W}
\end{equation}

\section{Implimentation using NAND gate}

\begin{equation}
    F(X,Y,Z,W) = \overline{\overline{ \overline{Y} . \overline{Z} +  X . \overline{Y} +  X . \overline{Z}. \overline{W} +  X . Z . W +  \overline{X} . Y . Z . \overline{W}}}
\end{equation}

\begin{equation}
    F(X,Y,Z,W) = \overline{  \overline{\overline{Y} . \overline{Z}} +  \overline{ X . \overline{Y} } +  \overline{ X . \overline{Z}. \overline{W} } +  \overline{ X . Z . W } +  \overline{ { \overline{X} . Y . Z . \overline{W} } } }
\end{equation}

\begin{comment}




\begin{table} [ht]
    \centering
    \begin{tabular}{ | p{1cm} | p{1cm} | p{1cm} | p{1cm} | p{2cm}  |}
    \hline
    X & Y & Z & W & F(X,Y,Z,W)\\ [0.5ex]
     \hline
     
    0 & 0 & 0 & 0 & 1\\
    0 & 0 & 0 & 1 & 1\\
    0 & 0 & 1 & 0 & 0\\
    0 & 0 & 1 & 1 & 0\\
    0 & 1 & 0 & 0 & 0\\
    0 & 1 & 0 & 1 & 0\\
    0 & 1 & 1 & 0 & 1\\
    0 & 1 & 1 & 1 & 0\\
    1 & 0 & 0 & 0 & 1\\
    1 & 0 & 0 & 1 & 1\\
    1 & 0 & 1 & 0 & 1\\
    1 & 0 & 1 & 1 & 1\\
    1 & 1 & 0 & 0 & 1\\
    1 & 1 & 0 & 1 & 0\\
    1 & 1 & 1 & 0 & 0\\
    1 & 1 & 1 & 1 & 1\\ [1ex]
    \hline
    \end{tabular}
    \label{fig1}
    \caption{Truth Table}
\end{table}
\end{comment}
\end{document}

